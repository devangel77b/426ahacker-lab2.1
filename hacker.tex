\documentclass[reprint,amsmath,amssymb,aps,twoside]{revtex4-2}


\usepackage{graphicx}
\usepackage{amsmath,amssymb,amsfonts}
\usepackage{dcolumn}
\usepackage{bm}
\usepackage{siunitx}
\sisetup{separate-uncertainty=true}
\usepackage[colorlinks,allcolors=blue]{hyperref}
\usepackage{cleveref}
\crefname{equation}{}{}
\crefname{figure}{Fig.}{Figs.}
\crefname{table}{Table}{Tables}
\usepackage{svg}
% set PDF metadata
\hypersetup{%
pdftitle={Testing Newton's second law with a cart-pulley system: a meta-analysis},
pdfauthor={Alyssa P.~Hacker},
}
\usepackage{fancyhdr}
\pagestyle{fancy}
\fancyhf{}
\fancyhead[RE,RO]{J S\&E \textbf{3}, xx--xx (2024)}
\fancyhead[LO]{Hacker}
\fancyhead[LE]{Testing Newton's second law: a meta-analysis}
\fancyfoot[C]{\thepage}
\fancypagestyle{mytitlepage}{
\fancyhf{}
\fancyhead[C]{Journal of Science \& Engineering \textbf{3}, xx--xx (2024)}
\fancyfoot[C]{\thepage}
}




\begin{document}
\setcounter{page}{1}
\title{Testing Newton's second law with a cart-pulley system: a meta-analysis}

\author{Alyssa P.~Hacker}
\email{Contact author: 426ahacker@frhsd.com}
\affiliation{Science \& Engineering Magnet Program, \href{https://manalapan.frhsd.com/}{Manalapan High School}, Englishtown, NJ 07726 USA}
\date{\today}

\begin{abstract}
Blah blah
\end{abstract}

\keywords{keywords here}

\maketitle\thispagestyle{mytitlepage}





\section{Introduction}
\begin{equation}
\sum\vec{F} = m\vec{a}
\label{eq:n2l}
\end{equation}

\begin{align}
T - m_1 g \mu &= m_1 a \\
m_2 g - T &= m_2 a 
\end{align}

\cite{tipler,barrons}
\begin{equation}
a = \dfrac{m_2}{m_1+m_2} g - \dfrac{m_1}{m_1+m_2} \mu g
\end{equation}

Constant force: \cite{arenas-2024-testing,avalur-2024-verifying,canada-2024-experimental,kishore-2024-relationship,yagnyeshwaran-2024-verifying}

Changing force: \cite{govardhanen-2024-newtons,kedharnath-2024-examining,krasnopolsky-2024-testing}

Did not vary: \cite{perle-2024-experimental}

No means to measure accceleration: \cite{barone-2024-investigating}

Exceed $g$ \cite{krasnopolsky-2024-testing}
Excessive friction or incorrect $m_1$ \cite{govardhanen-2024-newtons}, we use their data but with $m_1=\qty{2.5}{\kilo\gram}$.


Defining $\hat{a} = \dfrac{a}{g}$ and $\hat{m} = \dfrac{m_2}{m_1+m_2}$
\begin{equation}
\hat{a} = \hat{m} - (1-\hat{m}) \mu
\end{equation}
\begin{equation}
\hat{a} = (1+\mu) \hat{m} - \mu
\end{equation}
\begin{equation}
\hat{a} = 
\begin{cases}
(1+\mu) \hat{m} - \mu & \hat{m} \geq \dfrac{\mu}{1+\mu} \\
0 & \hat{m} < \dfrac{\mu}{1+\mu} 
\end{cases}
\end{equation}
The condition $\hat{m}=\dfrac{\mu}{1+\mu}$ is equivalent to $m_2=\mu m_1$; below this critical value static friction will keep the system from moving. 

For the case of no friction,
\begin{equation}
a = \dfrac{m_2}{m_1+m_2} g 
\label{eq:nofricdimensional}
\end{equation}
\begin{equation}
\hat{a} = \hat{m}
\label{eq:nofricdimensionless}
\end{equation}

\section{Methods and materials}
\cite{R-2024,ggplot2-2024}

\section{Results}
\begin{figure*}
\includesvg{fig2.svg}
\caption{Blah. Data from \cite{krasnopolsky-2024-testing} at highest two values of $m_2$ show timing issues with very short times and results that exceed the acceleration of gravity and so are removed from further analyses. Best fit line is $\hat{a}=0.948\hat{m}$ (linear regression, $R^2=0.9777$, $df=95$, $p<2e-16$), slope is not significantly different from 1.0 ($t$-test, $p=0.227$), providing support for \cref{eq:nofricdimensional,eq:nofricdimensionless,eq:n2l}.}
\end{figure*}

%\begin{figure*}
%\includesvg{fig3.svg}
%\caption{Blah. }
%\end{figure*}

\section{Discussion}
\section{Acknowledgements}
\bibliography{hacker.bib}
\end{document}
